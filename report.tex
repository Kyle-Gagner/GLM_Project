\documentclass[letterpaper,titlepage,10pt]{article}
\usepackage{graphicx}
\usepackage{amsmath}
\usepackage[letterpaper, portrait, margin=0.5in]{geometry}
\title{EE416 Final Project:\\Generalized Linear Models for Neural Data}
\date{Dec 7, 2018}
\author{Kyle Gagner and Fred Davis}
\begin{document}
\maketitle
\section{Abstract}
We did stuff
\section{Summary of Results}

\subsection{Simulating and Fitting Parameters to a GLM}

We used stimulus filter $f(t)=20e^{-t}$, self interaction filter $h(t)=-200e^{-t}$, and offset $b=-15$ as the
parameters to our GLM simulation. Filters were 15 coefficients long. The input stimulus was:

$$s(t)\sim
\begin{cases}
0 & t < 2000\\
\mathcal{N}(\mu=0.3, \sigma=0.1) & 2000 \leq t < 20000
\end{cases}$$

\begin{figure}[h]
\includegraphics[width=\textwidth]{section_4_fig1.pdf}
\caption{Simulation of a GLM and parameters fitted to the simulated data}
\label{fig_simulation}
\end{figure}

The filter coefficients (True series) and stimulus are ploted in Figure \ref{fig_simulation} along with the spiking
response of the simulation. Here it can be seen that spiking does not occur when the stimulus is zero because the
relatively large negative offset dominates. When the stimulus is noisy, spiking occurs. The spiking appears
qualitatively plausible.

A model was then fit to the simulated data. The coefficients and their standard error (Estimated series) are shown
overlaid on the ground truth in Figure \ref{fig_simulation}. The fit for the stimulus filter is good. The fit for the
self interaction filter is extremely poor for the first few coefficients and quickly improves, although it is not as
good as the fit to the stimulus filter. This can be explained by the fact that the optimization code depends on
observations of various interarrival times in order to estimate these coefficients. Due to the model's refractory
period, there are no or few observations for short interarrivals. By contrast, any spike in the response is a useful
observation for optimizing any of the stimulus filter coefficients, which is why their fits are relatively good.

\includegraphics[width=\textwidth]{section_4_fig2.pdf}\\
\includegraphics[width=\textwidth]{section_5_fig1.pdf}\\
\section{Analysis}
How we did the stuff
\section{References}
What other stuff we used to do the stuff
\section{Appendices}
Mention the code files that did the stuff
\end{document}
